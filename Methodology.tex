\chapter{Methodology}
\label{sec:methodology}
blindtext

\section{Datasets}
blindtext

\subsection{RailSem19 and used subsets}
blindtext

\subsection{vielleicht: Rail-DB for multi rail setection}
blindtext

\subsection{used annotations}
blindtext

\section{Switch evaluation dataset}
blindtext

\section{labeling task for temporal models}

\subsubsection{CVAT}
\subsubsection{Autolabler}
blindtext

\section{Setup used for Training CNNs}

% Hardware (GPU, CPU, CUDA)
For training \ac{CNN}s a powerful hardware setup is necessary.
In this work, one main setup is used to train all models.
This work is based on a project proposed and supported by the \ac{ICT} at the \ac{TU}, which also provides the necessary resources.
The department has a server with two NVIDIA Tesla V100S-PCIE-32GB \ac{GPU}s \cite{nvidia_v100_datasheet}.
Since the \ac{GPU}s utilized are equipped with 32 GB of memory and this is sufficient for the needs of this study, all trainings are done on a single \ac{GPU}.
No multi-\ac{GPU} setup is needed.
For this project, CUDA version 12.6 is used, which provides an environment for developing GPU-accelerated applications \cite{nvidia_cuda_126}.
The server employs two Intel(R) Xeon(R) Gold 5118 CPUs @ 2.30GHz \cite{intel_xeon_gold_prozessor_5118}.
These \ac{CPU}s have 24 threads and 12 cores each.

%Weights & Biases

All trainings are logged with the "Weights \& Biases" \cite{wandb}, a developer platform for training and fine-tuning machine learning models.
\cite{wandb} is utilized for logging configurations and results in this work.
The \textit{train loss} and \textit{validation loss} of each training is tracked and visualized in a graph.
The \textit{test \ac{IoU}} and the \textit{best validation loss} are displayed at the end of each training.
Additionally, the \ac{GPU}'s power usage and allocated memory are tracked and graphs are plotted.
Moreover, "Weights \& Biases" \cite{wandb} assigns a unique name to each training session, a critical feature when starting hundreds of training sessions.
All training logs are available at \href{https://wandb.ai/sebiorganization/train-ego-path-detection}{\texttt{https://wandb.ai/sebiorganization/train-ego-path-detection}}.

%PyTorch beschreiben

\section{Measuing Inference on NVIDIA Jetson Device}

Due to the nature of this work's use case and potential applications for the system encompassing safety features or preprocessing steps for autonomous trains, the system has to ensure real-time capability when deployed on embedded devices. 
In more detail, one goal for the train track prediction is to be deployed on an Nvidia Jetson device, because of their high computing power and their low power consumption.
Additionally, through NVIDIA's software ecosystem, rapid deployment and latency measurements are possible.
Jetson devices are suitable for applications in autonomous systems and computer vision tasks \cite{nvidia_jetson_embedded_devices}.
Furthermore, the NVIDIA Jetson series is specifically built for machine learning applications, because the devices have \ac{DLA}s built in.
\ac{DLA}s are tensor processor units designed to accelerate the inference of neuronal networks \cite{nvidia_dlas}.

\subsection{Hardware Setup for Measuring Inference}

For this study, the NVIDIA Jetson AGX Xavier is chosen because of several reasons.
This platform achieves up to 32 TOPS by utilizing a \ac{GPU} with 512 cores and 64 Tensor cores, which is advantageous for parallel data processing and neural network inference \cite{nvidia_jetson_agx_xavier_datasheet}.
An additional advantage present the two built-in \ac{NVDLA}s of the AGX Xavier.
\ac{NVDLA}s are NVIDIAs own \ac{DLA}s.
Even though there are more powerful devices like some of the NVIDIA Jetson Orin series, the AGX Xavier is sufficient for this application and cheaper \cite{nvidia_jetson_embedded_devices_prices}.
Additionally, the Xavier has a lower power consumption than the Orin.
Finally but yet important is the ease of integration.
Since, track prediction is a use case of a company, that most commonly uses the Jetson AGX Xavier platform.
Conducting tests on this specific device is appropriate.
The technical specifications of the NVIDIA Jetson AGX Xavier are shown in table \ref{tab:jetson_AGX_xavier_specs}.

\begin{table}[H]
    \centering
    \begin{tabular}{|l|l|}
    %\begin{tabular}{| p{0.3\linewidth} | p{0.6\linewidth} |}
        \hline
        AI Performance & 32 TOPS\\
        \hline
        \ac{GPU} & 512-core NVIDIA Volta GPU with 64 Tensor Cores\\
        \hline
        \ac{CPU} & 8-core NVIDIA Carmel ARM v8.2 64-bit CPU | 8 MB L2 + 4 MB L3\\
        \hline
        Memory & 32 GB 256-Bit LPDDR4x | 136.5 GB/s\\
        \hline
        Storage & 32 GB eMMC 5.1\\
        \hline
        DL Accelerator & (2x) NVDLA\\
        \hline
        Power & 10 W - 30 W\\
        \hline
    \end{tabular}
    \caption{Jetson AGX Xavier technical specifications \cite{nvidia_jetson_agx_xavier_datasheet}}
    \label{tab:jetson_AGX_xavier_specs}
\end{table}

\subsection{Optimizing models with TensorRT}

To fully leverage the used hardware platform Jetson AGX Xavier, NVIDIA introduced TensorRT \cite{nvidia_tensorrt}.
This ecosystem is developed to allow faster inference times when deploying deep learning models.
Also, the baseline paper \cite{tepNet2024} demonstrates through latency measurements that TensorRT consistently outperforms PyTorch in the context of speed.
TensorRT is approximately six times faster than PyTorch, which aligns with the claims made by NVIDIA \cite{tepNet2024} \cite{nvidia_tensorrt}.
Consequently, all latency measurements in this study are conducted using TensorRT.
This framework optimizes inference with methods like quantization, layer and tensor fusion, and kernel tuning \cite{nvidia_tensorrt}.
This can be done for various types of NVIDIA \ac{GPU}s.

\vspace{0.8cm}

\noindent\textbf{Quantization} can optimize an already trained model.
This technique shows a small reduction in accuracy but minimizes latency significantly.
While PyTorch uses PF32 for the inference of its standard models, TensorRT allows to use \ac{GPU}s and \ac{TPU}s to their maximum capacity by permitting FP8, INT8, and INT4.

\vspace{0.8cm}

\noindent\textbf{Layer and Tensor fusion} are used from TensorRT for further optimizing inference.
Often specific layer sequences include two consecutive layers that can be mathematically combined into a single layer.
Resulting in a reduction of unessential computations.

\vspace{0.8cm}

\noindent\textbf{Kernal tuning} is a process that seeks the optimal configuration of available kernels in the Jetson device.
The selected kernels depend on the specific machine-learning application and the used device.
In more detail, the model is executed on the device several times using different CUDA kernels in each run and the best combination is utilized.
Since, Kernel tuning is an iterative process it usually takes a couple of minutes even with compact models.

\subsubsection{TensorRT engine from PyTorch model}

Since TensorRT does not support PyTorch models, a workaround has to be made.
First, the PyTorch models are converted into an \ac{ONNX} format \cite{onnx_docu}.
After that, the \texttt{.onnx} files are optimized by TensorRT with the techniques mentioned before.

\vspace{0.8cm}

\ac{ONNX} is an approach for easier access to hardware optimization and to make interoperability possible \cite{onnx_docu}.
Often machine learning applications are locked in the framework they are developed in, which can present some hurdles.
The \ac{ONNX} format aims for a standardized representation of machine learning models and is therefore commonly used in the community.
Once converted to ONNX, a model utilizes standard data types and a set of built-in operators.
The \ac{ONNX} format version, known as the "opset", defines which operators are used \cite{onnx_docu}.
Therefore the TensorRT version must be compatible with the \ac{ONNX} opset number.

\vspace{0.8cm}

The TensorRT version installed on the Jetson AGX Xavier must support the used \ac{ONNX} opset.
Therefore in this work, the opset 11 is used for all model exports.

\begin{listing}[H]
\begin{minted}[
    frame=single,
    framesep=2mm,
    baselinestretch=1.2,
    bgcolor=white,
    fontsize=\footnotesize,
    linenos
    ]{python}
# Export the model to ONNX
torch.onnx.export(
    model,                       # Model to be exported
    input_tensor,                # Input to the model
    "onnx_file_path/model.onnx", # Output file path
    opset_version=11,            # ONNX version to export the model to
    export_params=True,          # Store the trained parameter
                                 # weights inside the model file
)
\end{minted}
\caption{Exporting a PyTorch model to \ac{ONNX} format}
\label{code:export_model_onnx}
\end{listing}

In this work, all PyTorch models are converted to \texttt{.onnx} files with the built-in torch exporter.
\autoref{code:export_model_onnx} shows that the conversion is done with a single \texttt{torch.onnx.export()} python line \cite{pytorch_onnx_exporter_docu}.
After an \ac{ONNX} model format is created it can be optimized by TensorRT.
This can be executed with the \texttt{trtexec} console application.

\vspace{0.5cm}
\begin{center}
    \texttt{trtexec --onnx=model.onnx --saveEngine=model.engine}
\end{center}
\vspace{0.5cm}

This command provides an example, in which the model.onnx is optimized with TensorRT techniques mentioned above.
Furthermore, the \texttt{model.engine} is saved and can be executed from now on without the need to create it again.
After a couple of minutes, TensorRT outputs a performance summary with many different latencies, like the min, max, mean, or median.
Of these values, the median inference time is considered as the final latency rather than the mean, because it is more robust against outliers.