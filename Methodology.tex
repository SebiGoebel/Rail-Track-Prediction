\chapter{Methodology}
\label{sec:methodology}
blindtext

\section{Datasets}
blindtext

\subsection{RailSem19 and used subsets}
blindtext

\subsection{vielleicht: Rail-DB for multi rail setection}
blindtext

\subsection{used annotations}
blindtext

\section{Switch evaluation dataset}
blindtext

\section{labeling task for temporal models}

\subsubsection{CVAT}
\subsubsection{Autolabler}
blindtext

\section{Setup used for Training CNNs}

% Hardware (GPU, CPU, CUDA)
For training \ac{CNN}s a powerful hardware setup is necessary.
In this work, one main setup is used to train all models.
This work is based on a project proposed and supported by the \ac{ICT} at the \ac{TU}, which also provides the necessary resources.
The department has a server with two NVIDIA Tesla V100S-PCIE-32GB \ac{GPU}s \cite{nvidia_v100_datasheet}.
Since the \ac{GPU}s utilized are equipped with 32 GB of memory and this is sufficient for the needs of this study, all trainings are done on a single \ac{GPU}.
No multi-\ac{GPU} setup is needed.
For this project, CUDA version 12.6 is used, which provides an environment for developing GPU-accelerated applications \cite{nvidia_cuda_126}.
The server employs two Intel(R) Xeon(R) Gold 5118 CPUs @ 2.30GHz \cite{intel_xeon_gold_prozessor_5118}.
These \ac{CPU}s have 24 threads and 12 cores each.

%Weights & Biases

All trainings are logged with the "Weights \& Biases" \cite{wandb}, a developer platform for training and fine-tuning machine learning models.
\cite{wandb} is utilized for logging configurations and results in this work.
The \textit{train loss} and \textit{validation loss} of each training is tracked and visualized in a graph.
The \textit{test \ac{IoU}} and the \textit{best validation loss} are displayed at the end of each training.
Additionally, the \ac{GPU}'s power usage and allocated memory are tracked and graphs are plotted.
Moreover, "Weights \& Biases" \cite{wandb} assigns a unique name to each training session, a critical feature when starting hundreds of training sessions.
All training logs are available at \href{https://wandb.ai/sebiorganization/train-ego-path-detection}{\texttt{https://wandb.ai/sebiorganization/train-ego-path-detection}}.

%PyTorch beschreiben

\section{Measuing Inference on NVIDIA Jetson Devices}
blindtext

\subsection{The NVIDIA Jetson series}


%Quelle: Jetson AGX Xavier Series
% https://www.nvidia.com/en-us/autonomous-machines/embedded-systems/jetson-agx-xavier/

The technical specifications of the NVIDIA Jetson AGX Xavier is shown in table \ref{tab:jetson_AGX_xavier_specs}.

\begin{table}[H]
    \centering
    \begin{tabular}{|l|l|}
    %\begin{tabular}{| p{0.3\linewidth} | p{0.6\linewidth} |}
        \hline
        AI Performance & 32 TOPS\\
        \hline
        \ac{GPU} & 512-core NVIDIA Volta GPU with 64 Tensor Cores\\
        \hline
        \ac{CPU} & 8-core NVIDIA Carmel ARM v8.2 64-bit CPU | 8 MB L2 + 4 MB L3\\
        \hline
        Memory & 32 GB 256-Bit LPDDR4x | 136.5 GB/s\\
        \hline
        Storage & 32 GB eMMC 5.1\\
        \hline
        DL Accelerator & (2x) NVDLA\\
        \hline
        Power & 10 W - 30 W\\
        \hline
    \end{tabular}
    \caption{Jetson AGX Xavier technical specifications \cite{nvidia_jetson_agx_xavier_datasheet}}
    \label{tab:jetson_AGX_xavier_specs}
\end{table}

\subsection{Optimizing models through TensorRT}
blindtext
