\section{improved TEP-Net}

Various adaptations improve the performance of the model.
Experiments always consider one aspect of the architecture.
Then, the best model is used to improve performance further.
First experiments utilize different backbones.
After evaluating models' performances with various backbone versions, one is selected based on the best combination of accuracy and speed for further experiments.
The next stage applies pooling layers.
The best-performing model is then equipped with different prediction heads, resulting in models of which one again outperforms its competitors.
This model defines the final model architecture.

All trainings of different architectures utilize the same hyperparameters to compare resulting models fairly.
Numerous parameters from \cite{tepNet2024} are utilized as a basis for the guideline.
All experiments, which include adaptations of the baseline model, are pre-trained on COCO and fine-tuned on the RailSem19 dataset with the annotations of \cite{tepNet2024}.
Images of this dataset are randomly assigned to one of three subdatasets for training, validation, and testing with proportions of $80\%-10\%-10\%$.
For reproducibility of experiments, a seed of 42 is given.
Since models are solely trained on image crops, the size of these crops is set to $3 \times 512 \times 512$, which is nearest to crops from practical use cases.
Furthermore, the Adam optimizer \cite{pytorchAdamOptimizer}, in combination with the OneCycle scheduler \cite{pytorch_oneCycleLR_docu} from PyTorch, is used for training models.
First, the scheduler increases the learning rate to 0.0001 and then reduces it.
Models are all trained with a batch size of 8 for 400 epochs.

Additionally, the settings for data augmentation stay the same for all experiments with various models.
The image variation settings include factors for brightness, contrast, and saturation of 0.5.
The hue factor is set to 0.2.
These values are inspired by \cite{tepNet2024}.
However, changes in this work have been done for cropping parameters.
The margins to the top and the sides are increased to 0.2.
The standard deviation factors for the sides and the top are set to 0.3 and 0.1, respectively.

\subsection{Backbones}

In this work, experiments with four different backbones are conducted.
\cite{tepNet2024} has experimented with various ResNet and EfficienNets versions.
Literature shows that MobileNets and DenseNets have also introduced significant advancements.
The expected results of tests with DenseNets are high accuracies.
When training models that incorporate MobileNets, an advantage in latency speed is expected;
therefore, all versions of those two network architectures are implemented that are available in PyTorch \cite{pytorchmobilenetv3} \cite{pytorchdensenet}.
\autoref{tab:backboneVersions} overviews all backbones and their versions utilized in this work.
Additionally, section \autoref{subsubsec:backbones} explains the used layers.
Since all backbones must be interchangeable, this work discards some backbone layers.

\subsection{Pooling Layers}

This work either uses an adaptive average pool or an adaptive max pool to transform the output tensor of the backbone.
After reducing this tensor to a defined channel size with a \textit{conv2D1x1} layer, the pooling layers are responsible for additionally encoding spatial features into a vector-like tensor with dimensions of $C \times 1 \times 1$.
\autoref{subsubsec:pooling} gives a detailed description of the pooling layer's implementation.

\subsection{Prediction Heads}

Furthermore, experiments include various prediction heads.
This work implements a depth-head, a width-head, and a trapeze-head.
These parts of the architecture differ in number of \ac{FC} layers and the size of these layers.
All of them result in an output vector with a pre-defined size.
A size of 129 ($2 \times H + 1$ with $H = 64$) was predominantly utilized.

\vspace{1cm}

Additional experiments include an input size of $3 \times 224 \times 224$, batch normalization before each ReLu activation function, and various numbers of anchors like 257 ($2 \times H + 1$ with $H = 128$).
A performance gain is strived for in each of those experiments.
The reduced image size is chosen because all four papers of the backbones recommend this size.
The increased number of anchors should give a more detailed description of the predicted rail track.