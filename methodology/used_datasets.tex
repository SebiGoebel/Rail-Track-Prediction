\section{Datasets}

\begin{itemize}
    \item RailSem19 for Yolos + used subsets [1/2 bis 1 Seite] fertig
    \item TEP-annotations [1/2 Seite bzw. ein kurzer Absatz] fertig
    \item Switch evaluation dataset (all images with switches from val and test dataset) [1/2 Seite bzw. ein kurzer Absatz]
    \item auto labler + temporal dataset + dass ich CVAT versendet habe[1 bis 2 Seiten]
\end{itemize}

\subsection{RailSem19 and subsets for training object detection models}

The first approach to creating a Rail Track Prediction system includes combining an object detection model with a semantic segmentation model.
Since RailSem19 supports both of these methods this dataset is chosen for training.
Furthermore, state-of-the-art research in \autoref{sec:datasets} shows that this dataset is the only one that not only includes switch annotations but also labels for switch states: $switch\_left$, $switch\_right$.
Since not all states are identifiable even though a switch is visible there also is the $switch\_unknwon$ label.

In this work, the bounding boxes are utilized to train the \ac{YOLO} object detection models.
Experiments have been conducted with three versions of this dataset.
An overview of these subsets is shown in \autoref{tab:usedSubsetsforYOLOs}.
The first set is the whole dataset, which includes 8500 images with all different bounding box labels.
The second subset only considers images with switch labels including the $switch\_unknown$ label.
For this subset, 2764 images remain. The third subset only considers the switch labels in which the state is identifiable.
This is because when focusing on the train's direction, the $switch\_unknown$ label does not hold any valuable information.
To prevent confusion, all images containing $switch\_unknown$ annotations are intentionally excluded, resulting in a final set of 1,240 images.

\begin{table}[H]
    \centering
    \begin{tabular}{|l|l|l|}
    %\begin{tabular}{| p{0.3\linewidth} | p{0.6\linewidth} |}
        \hline
        \textbf{RailSem19} & \textbf{RailSem19\_onlySwitches} & \textbf{RailSem19\_onlySwitchesLeftRight}\\
        \hline
        8500 images & 2764 images & 1240 images\\
        \hline
        all bounding box labels & $switch\_left$ & $switch\_left$\\
        \hline
        & $switch\_right$ & $switch\_right$\\
        \hline
        & $switch\_unknown$ & images with $switch\_unknown$ excluded\\
        \hline
    \end{tabular}
    \caption{Used dataset subsets of RailSem19 for training \ac{YOLO} object detection models}
    \label{tab:usedSubsetsforYOLOs}
\end{table}

Since the experiments of training object detection models showed unsatisfactory results also described in <section>, this methodology was deemed ineffective.
This leads to the pursuit of a different solution and experiments with semantic segmentation models are therefore not accomplished.
Consequently the corresponding dense labels of the RailSem19 dataset are not utilized.

\clearpage
\subsection{TEP annotations}

For training single-frame-based models \cite{tepNet2024} published its annotations for the images of RailSem19.
These annotations consist of polylines for the left and the right rail of the train.
Only the two rails are included in the labels which the train drives on.
This is also the case when switches are present.
Additionally, some images are excluded from this dataset when the train's track is not identifiable.
For labeling this dataset the online tool CVAT \cite{cvat} is used.
These annotations can then be transformed with pre processing step to support other models like semantic segmentation. 
This dataset is described in more detail in \autoref{subsubsec:TEP-Net_dataset}.

\subsection{Switch evaluation dataset}

blindtext

\subsection{Temporal Dataset}

blindtext
