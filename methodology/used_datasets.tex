\section{Datasets}

\begin{itemize}
    \item RailSem19 for Yolos + used subsets [1/2 bis 1 Seite] fertig
    \item TEP-annotations [1/2 Seite bzw. ein kurzer Absatz] fertig
    \item Switch evaluation dataset (all images with switches from val and test dataset) [1/2 Seite bzw. ein kurzer Absatz]
    \item auto labler + temporal dataset + dass ich CVAT versendet habe[1 bis 2 Seiten]
\end{itemize}

\subsection{RailSem19 and subsets for training object detection models}

The first approach to creating a Rail Track Prediction system includes combining an object detection model with a semantic segmentation model.
Since RailSem19 supports both of these methods this dataset is chosen for training.
Furthermore, state-of-the-art research in \autoref{sec:datasets} shows that this dataset is the only one that not only includes switch annotations but also labels for switch states: $switch\_left$, $switch\_right$.
Since not all states are identifiable even though a switch is visible there also is the $switch\_unknwon$ label.

In this work, the bounding boxes are utilized to train the \ac{YOLO} object detection models.
Experiments have been conducted with three versions of this dataset.
An overview of these subsets is shown in \autoref{tab:usedSubsetsforYOLOs}.
The first set is the whole dataset, which includes 8500 images with all different bounding box labels.
The second subset only considers images with switch labels including the $switch\_unknown$ label.
For this subset, 2764 images remain. The third subset only considers the switch labels in which the state is identifiable.
This is because when focusing on the train's direction, the $switch\_unknown$ label does not hold any valuable information.
To prevent confusion, all images containing $switch\_unknown$ annotations are intentionally excluded, resulting in a final set of 1,240 images.

\begin{table}[H]
    \centering
    \begin{tabular}{|l|l|l|}
    %\begin{tabular}{| p{0.3\linewidth} | p{0.6\linewidth} |}
        \hline
        \textbf{RailSem19} & \textbf{RailSem19\_onlySwitches} & \textbf{RailSem19\_onlySwitchesLeftRight}\\
        \hline
        8500 images & 2764 images & 1240 images\\
        \hline
        all bounding box labels & $switch\_left$ & $switch\_left$\\
        \hline
        & $switch\_right$ & $switch\_right$\\
        \hline
        & $switch\_unknown$ & images with $switch\_unknown$ excluded\\
        \hline
    \end{tabular}
    \caption{Used dataset subsets of RailSem19 for training \ac{YOLO} object detection models}
    \label{tab:usedSubsetsforYOLOs}
\end{table}

Since the experiments of training object detection models showed unsatisfactory results also described in <section>, this methodology was deemed ineffective.
This leads to the pursuit of a different solution and experiments with semantic segmentation models are therefore not accomplished.
Consequently the corresponding dense labels of the RailSem19 dataset are not utilized.

\clearpage
\subsection{TEP annotations}

For training single-frame-based models \cite{tepNet2024} published its annotations for the images of RailSem19.
These annotations consist of polylines for the left and the right rail of the train.
Only the two rails are included in the labels which the train drives on.
This is also the case when switches are present.
Additionally, some images are excluded from this dataset when the train's track is not identifiable.
For labeling this dataset the online tool CVAT \cite{cvat} is used.
These annotations can then be transformed with pre processing step to support other models like semantic segmentation. 
This dataset is described in more detail in \autoref{subsubsec:TEP-Net_dataset}.

\subsection{Switch evaluation dataset}

For a practical application, it is important to make the output as useful and visible as possible.
Therefore, the output should be the whole track, not only the rails.
This is realized in the post-processing by filling out the area between the rails resulting in a binary mask.
This mask is similar to the output of semantic segmentation techniques.
It gives each pixel either the class label $track$ or $no\_track$.
For these methods, the most common evaluation metric is the \ac{IoU}, which is also used by \cite{tepNet2024} to evaluate models on the test set.

This metric has its justification for general rail track prediction scenarios because it is a good indicator of model performance.
However, when switches are present in the scene the \ac{IoU} loses meaningfulness.
This issue is visualized in <Figure xy>.
In this example, the model cannot correctly predict the train's path at the switch.
Contrary to this uncertainty, the \ac{IoU} still gives a high value indicating a good performance even though the direction of the train is wrong.
The reason for this lies in the calculation of the \ac{IoU} and this specific problem case.
Since the switch is in the distance and the track is mostly correct up to this point, the predicted mask and the mask of the \ac{GT} still share a great portion of their areas.
This leads to high values of the \ac{IoU} metric, even though the track is incorrect.

\vspace{1cm}
<Bild wo ich zeige dass die IoU nicht sehr aussage kräftig ist wenn es um weichen geht>
\vspace{1cm}

Since this work especially focuses on correctly predicting the train's direction in scenarios with switches, a solution to this evaluation issue must be found.
Therefore a switch evaluation dataset is created, which solely focuses on switch scenarios.
Furthermore, a points system is created that gives insight into the model performance.
This point system is shown in <Figure xy>.
The dataset consists of all images that contain switch cases out of the validation and test dataset that are obtained from the $80\%-10\%-10\%$ split of \cite{tepNet2024}.
This results in 67 images which include 72 switches in total.
The model gets 1 point for each correctly predicted switch.
There are cases where models correctly filter out the track and set the horizon line before the switch.
This behavior is rewarded with 0,5 points.
This behavior is assumed to come from the labeling policy with $switch\_unknown$ labels.
In \cite{tepNet2024} in images with $switch\_unknown$ labels, the track is annotated up to this label.
This shows that the model is uncertain about the switch state.
However, setting the horizon line before the switch is preferred over predicting the track incorrectly.
In other cases, the model does not get any points.

\vspace{1cm}
<Bild vom punkte system, siehe präsi>
\vspace{1cm}

This switch dataset deals with a qualitative performance evaluation and no technique is found to automate this process.
Therefore, a model predicts the track of each image, and the points are given by manually observing outputs.
Since the prediction process should resemble a real application, the autocrop technique is utilized.
By predicting each image 50 times crop coordinates have time to converge to a crop similar to one in real applications.

\subsection{Temporal Dataset}

blindtext
