\chapter{Introduction}
\label{sec:introduction}

\begin{itemize}
    \item Worum geht es?
    \item Was mach ich und warum?
    \item Motivation
    \item Problem
    \item Ziel
\end{itemize}




Autonomous vehicles are increasingly considered a groundbreaking technology of the future. \cite{FraunhoferInstituteforCognitiveSystemsIKS.21.10.2024}

\vspace{2cm} % Größerer Abstand zwischen den Reihen

\textbf{Bild}

\vspace{2cm} % Größerer Abstand zwischen den Reihen

\textbf{\ref{sec:stateOfTheArt}} \textbf{\nameref{sec:stateOfTheArt}} describes different approaches, to how the train track prediction problem can be solved.
In this section, thorough research is done, which gives an overview of different approaches, models, and fitting datasets.
Additionally, a couple of papers that could serve as a baseline are discussed in more detail.
In \textbf{\ref{sec:methodology}} \textbf{\nameref{sec:methodology}} the used datasets as well as the hardware and software frameworks used for training and evaluation are described.
Additionally, this work includes the development of a temporal dataset with partly auto-labeled annotations.
The labeling process and algorithms for auto-labeling are also described in this section.
After that carried-out experiments are described in \textbf{\ref{sec:experiments}} \textbf{\nameref{sec:experiments}} and their results are described in \textbf{\ref{sec:results}} \textbf{\nameref{sec:results}} following a discussion in \textbf{\ref{sec:discussion}} \textbf{\nameref{sec:discussion}}.
In the final section \textbf{\ref{sec:conclusionAndOutlook}} \textbf{\nameref{sec:conclusionAndOutlook}}, the work is summarized and further possible ideas for improvements are presented.

\section{Erste Überschrift Tiefe 2 (section)}
blindtext

\subsection{Erste Überschrift Tiefe 3 (subsection)}
blindtext

\subsubsection{Erste Überschrift Tiefe 4 (subsubsection)}
blindtext
\cite{orf2024toetlicherZugunfall}