\section{Semantic Segmentation}
\label{sec:SemanticSegmentation}

The first approach to filter out the rail tracks in front of the train was proposed by <Quelle: 2018: Efficient Rail Area Detection Using Convolutional Neural Network [in TEP]> in 2018.
A SegNet <Quelle: SegNet> inspired network for semantic segmentation is extended with mixed pooling <Quelle> and atrous spatial pyramid pooling (ASPP) from DeepLab.
After the proposed semantic segmentation network outputs a binary mask with pixel labels being track or no track, a polygon fitting technique is utilized to refine the tracks further.

In 2019 <RailNet: A Segmentation Network for Railroad Detection> introduced the RailNet architecture.
This model uses a ResNet-50 <Quelle ResNet> backbone and a fully convolutional network <Quelle in TEP-Net> to segment the rail tracks.
The network is designed in a pyramid structure <Quelle: 2017 Feature pyramid networks for object detection>, in which features of every ResNet stage are summed and up-sampled.
This combines low and high-level features, resulting in an enhanced performance.
<Quelle: RailNet: A Segmentation Network for Railroad Detection> reports higher accuracy than state-of-the-art segmentation models at the time with speeds up to 20 FPS.
Tests are made on the introduced RSDS dataset further described in <Dataset section>.

In 2019 RailSem19 <Quelle: RailSem19> introduced the first publicly available dataset for the rail domain.
This dataset also includes annotations for semantic segmentation tasks and experimented with a FRRNB model <Quelle: Full-Resolution Residual Networks for Semantic Segmentation in Street Scenes>.
This dataset is widely used in the community when working in the rail domain.

<Quelle: 2020 RailNet: An Information Aggregation Network for Rail Track Segmentation> already uses a subset of RailSem19 in 2020.
Another RailNet model is proposed that uses a VGG16 backbone and an Information Aggregation Module.
<Quelle: 2020 RailNet: An Information Aggregation Network for Rail Track Segmentation> only predicts the rails in RailSem19, other annotations are ignored.
The characteristics of rails like placement and structure are considered.
Therefore the integrated module is implemented to improve spatial relationship between features on both the vertical and horizontal axes.

<Quelle: 2022 Automated Semantic Segmentation for Autonomous Railway Vehicles> also used RailSem19.
The U-Net <Quelle: U-Net> architecture is trained on four different subsets of RailSem19, including 2, 3, 4, or all 19 classes.
Additionally, A cropping method is implemented to account for the difference in resolutions between the recommended one for U-Net and one of RailSem19's images.

<Quelle: 2022 Application of Rail Segmentation in the Monitoring of Autonomous Train’s Frontal Environment> used the RailSet dataset <Quelle: 2022 RailSet: A Unique Dataset for Railway Anomaly Detection>, which is an extension of RailSem19 for segmentation and anomaly detection tasks.
For details of the RailSet dataset please refer to <dataset section>.
<Quelle: 2022 Application of Rail Segmentation in the Monitoring of Autonomous Train’s Frontal Environment> trained U-Net <Quelle: U-Net> and FRNN <Quelle: FRNN> and incorporated horizontal flips and zooming into the data augmentation.

In 2020 <Quelle: Railroad semantic segmentation on high-resolution images> proposed a U-Net-inspired <Quelle: U-Net> semantic segmentation network.
It combines a ResNet-34 backbone and includes connections to the upsampling blocks.
At the deepest level, a spatial pyramid pooling (SPP) <Quelle: Spatial pyramid pooling in deep convolutional networks for visual recognition> and on the skip connections squeeze-and-excitation blocks <Quelle: wie bei mobilenet> are used.
Tests on RailSem19 <Quelle: RailSem19>, show that it outperforms <Quelle: 2019 RailNet: A Segmentation Network for Railroad Detection> in accuracy with a speed of 20 FPS.
Additionally, <Quelle: Railroad semantic segmentation on high-resolution images> introduced the concept of "possible tracks", which are all paths the train could follow under the assumption that the state of switches cannot be determined.
For this, a rule-based post-processing algorithm is proposed. <Figure xy> shows the concept of possible paths.

\vspace{1cm}

<vielleicht figure possible tracks von Quelle: Railroad semantic segmentation on high-resolution images>

\vspace{1cm}

In 2023 <Quelle: TPE-Net: Track Point Extraction and Association Network for Rail Path Proposal Generation> further investigated the task of possible tracks and introduced Track Point Extraction and Association Network (TPE-Net).
It consists of a DenseNet-inspired architecture, which is trained and tested on RailSem19 and outputs regressed heatmaps besides the segmented rails.
An example of such a heatmap is an image with one channel that shows the probability of each pixel being within a rail track.
The segmentation mask and the heatmaps are then used by a complex post-processing approach.
This process includes track point clustering, the creation of track segments, and the creation of a path tree, which is then used to generate all possible tracks.
Refinement is done by removing redundant paths and polynomial fitting.
Because of the complexity of this system, <Quelle: TPE-Net: Track Point Extraction and Association Network for Rail Path Proposal Generation> reports speeds up to 12 FPS making this system unsuitable for real-time applications.
Additionally, it is stated that problems arise when switches are present.

In 2022 <Quelle: RailVID: A Dataset for Rail Environment Semantic> proposed the RailVID dataset, which consists of infrared images instead of RGB data to improve the system's abilities in challenging situations such as the absence of ambient light.
The dataset is described in <Dataset section> in more detail.
After collecting data, <Quelle: RailVID: A Dataset for Rail Environment Semantic> experimented with widely used CNNs for semantic segmentations, like CGNet <Quelle>, DeepLabv3+ <Quelle>, and BiSeNet <Quelle>.
Additionally, an improved BiSeNet architecture is proposed with consideration of the infrared data. Performance is enhanced by added layers to fuse low-level features.

In 2021 <Accurate and Lightweight RailNet for Real-Time Rail Line Detection> proposed another architecture called RailNet.
It consists of an Encoder-Decoder structure incorporating depth-wise convolutions and a Segmentation Soul block, inspired by the context embedding block of BiSeNetV2 <Quelle: BiSeNetV2>.
Additionally, a port processing algorithm is utilized that is based on sliding window detection.
<Accurate and Lightweight RailNet for Real-Time Rail Line Detection> reports speeds up to 74 FPS proving that this system is real-time capable.

A topic that could be interesting is anomaly detection because many state-of-the-art approaches use semantic segmentation as a preprocessing step.
<2021 Near Real-time Situation Awareness and Anomaly Detection for Complex Railway Environment> utilizes a BiSeNet architecture for segmenting rails, which is tailored for the anomaly detection task.
This network can detect small objects on the track.
Therefore accuracy is preferred over speed, resulting in a system that is not real-time capable.

\subsection{Example Überschrift}

blindtext

\subsection{Example Überschrift}

blindtext hallo